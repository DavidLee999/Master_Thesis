\chapter{Theoretical background of thermal infrared remote sensing}

\label{Chapter2}

%description
%----------------------------------------------------------------------------------------
This chapter reviews some fundamentals on the thermal remote sensing (Section 2.1), as well as a useful approach, which is also used in the MITIP method, to identify and determine target temperature of subpixle resolution (Section 2.2). \\

%----------------------------------------------------------------------------------------
%	SECTION 1
%----------------------------------------------------------------------------------------

\section{Principles of thermal infrared remote sensing}
Thermal remote sensing depends on the fact that any object with a temperature above absolute zero (0 K or -273.15 $^\circ$C) emits electromagnetic (EM) radiation in the infrared range. For example, the Earth we live has an average temperature around 300 K and its maximum emittance falls on thermal infrared (TIR) domain \parencite {Reference201}. The spectral intensity and composition of the emited radiation are determined by its surface temperature, which is also called kinetic temperature $T_{kin}$,  and the emissivity of the object. The emissivity $\epsilon_{\lambda}$, $\epsilon$ for short, is a ratio of the radiant flux of an object at a certain temperature to teh radiant flux of a blackbody at the same temperature. The blackbody is an idealized physical object that aborbs all incident EM radiation, which means its emissivity is 1. The emissivity varies as a function of wavelength $\lambda$ and also depends on the surface type of the object, but is not temperature-dependent \parencite{Reference202, Reference203}.\\

The satellite remote sensing snesors responsive in te thermal infrared domain are capable to record the EM radiations emited by Earth surface objects. Thus these sensors will produce thermal radiance images which record the equivalent blackbody radiances of the object on the Eearth's surface. From that the derivation of the so-called radiance temperature, or brightness temperature, $T_{rad}$ is possible \parencite{Reference204, Reference205}. Radiance temperature $T_{rad}$ is the radiance measured at the sensor in terms of temperature of a equivalent blackbody \parencite{Reference206}. Notice that the brightness temperature $T_b$ and the kinematic temperature $T_{kin}$ are two different terms and the conversion between them will be introduced later.
%-----------------------------------
%	SUBSECTION 1
%-----------------------------------

\subsection{The thermal infrared  domain and atmospheric windows}

%-----------------------------------
%	SUBSECTION 2
%-----------------------------------

\subsection{The Planck's law and Stefan-Boltzmann law}

%----------------------------------------------------------------------------------------
%	SECTION 2
%----------------------------------------------------------------------------------------

\section{A dual-channel method for the identification of subresolution high temperature sources}

%----------------------------------------------------------------------------------------
%	SECTION 3
%----------------------------------------------------------------------------------------

%\section{The mitip, an atmospheric correctoin and image processing method}